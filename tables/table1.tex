\begin{table}[t!]
\begin{adjustwidth}{0in}{0in} % comment out/remove adjustwidth environment if table fits in text column.
\centering
\begin{tabular}{llll}
\hline
{\bf Provider Network} & {\bf Responsive URLs} & {\bf Stable URLs*} & {\bf Reliable URLs} \\ \hline
BHL\textsuperscript{a} & 29.99\% (77,040) & 99.95\% (241,243) & 29.97\% (76,998) \\
DataONE\textsuperscript{b} & 92.54\% (394,568) & 87.11\% (367,957) & 80.30\% (342,363) \\
GBIF\textsuperscript{c} & 73.93\% (60,564) & 33.93\% (22,491) & 24.53\% (20,093) \\
iDigBio\textsuperscript{c} & 86.80\% (5,988) & 61.99\% (4,265) & 54.41\% (3,754)  \\
All observed URLs** & 69.62\% (534,107) & 86.46\% (632,879) & 57.43\% (440,606) \\ \hline
\end{tabular}
\caption{Overall responsiveness, stability, and reliability for URLs observed in each aggregator's provider network and for all observed provider network URLs as of May 2020. Numbers in brackets indicate total URL counts.
*URLs that never provided content were omitted from the denominator when calculating Stable URLs percentages.
**Because URLs may be registered in more than one provider network, the total number of observed URLs is expected to be less than the sum of the URL counts for each network.
\textsuperscript{a}\citet{poelen_zenodo_bhl}
\textsuperscript{b}\citet{poelen_zenodo_dataone}
\textsuperscript{c}\citet{poelen_zenodo_gib}
}
\label{table1}
\end{adjustwidth}
\end{table}
